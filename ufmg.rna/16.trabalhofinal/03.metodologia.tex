Para avaliar as duas funções a seguinte metodologia foi utilizada:

\begin{enumerate}
	\item $85\%$ dos dados disponíveis foram utilizados para treinar a rede;
	\item Após o treinamento, os $15\%$ dos dados restantes foram avaliados pela rede neural criada;
	\item A confusion matrix dos resultados avaliados foi calculada;
	\item O índice de acertividade foi calculado segundo a fórmula:
	$$ \text{índice acertividade} = \frac{\text{soma diagonal principal}}{\text{soma toda matriz}} $$
	\item Os passos 1 a 4 foram repetidos 10 vezes para cada método e cada base de dados;
	\item A média do índice de acertividade foi então calculada.
\end{enumerate}

Para executar os redes foi utilizado o método \texttt{mlp} do pacote \texttt{RSNNS}. Nesse método o parâmetro \texttt{size} diz respeito à quantidade de neurônios na rede neural. O parâmetro \texttt{learnFuncParams} define paramêtros para as funções de aprendizado. E o parâmetro \texttt{maxtit} define a quantidade máxima de épocas para a etapa de aprendizagem \cite{bib-rsnns}.

As bases de dados utilizadas estão apresentadas a seguir.

\subsection{Iris}

O primeiro problema estudado foi o \emph{Iris} \cite{bib-iris}.

\begin{itemize}	
	\item \textbf{Número de instâncias}: 150
	\item \textbf{Número de atributos}: 4
	\item \textbf{Número de classes}: 3
\end{itemize}

Essa é uma base de dados muito conhecida no ramo de Redes Neurais Artificias. Ela avalia dimensões físicas de espécimes de três classes de plantas iris: Iris Setosa, Iris Versicolour e Iris Virginica.

\subsection{Letter Recognition}

O segundo problema estudado foi o \emph{Letter Recognition} \cite{bib-letter-recognition}.

\begin{itemize}	
	\item \textbf{Número de instâncias}: 20000
	\item \textbf{Número de atributos}: 16
	\item \textbf{Número de classes}: 26
\end{itemize}

Nesse problema as 26 letras do alfabeto inglês são identifacadas em imagens preto e branco de onde foram extraídos 16 atributos numéricos.

\subsection{Wine}

O terceiro problema estudado foi o \emph{Wine} \cite{bib-wine}.

\begin{itemize}	
	\item \textbf{Número de instâncias}: 178
	\item \textbf{Número de atributos}: 13
	\item \textbf{Número de classes}: 3
\end{itemize}

Nesse problema resultados de analíses químicias são usados para determinar a origem de vinhos.

\subsection{Tic Tac Toe}

O quarto problema estudado foi o \emph{Tic Tac Toe} \cite{bib-tic-tac-toe}.

\begin{itemize}	
	\item \textbf{Número de instâncias}: 958
	\item \textbf{Número de atributos}: 9	
	\item \textbf{Número de classes}: 2
\end{itemize}

Nesse problema, todas as possibilidade de um tabuleiro do \emph{Jogo da Velha} onde $x$ começou são mostradas e classificadas quanto à possibilidade do $x$ vencer ou não.

\subsection{Balance Scale}

O quinto problema estudado foi o \emph{Balance Scale} \cite{bib-balance-scale}.

\begin{itemize}	
	\item \textbf{Número de instâncias}: 625
	\item \textbf{Número de atributos}: 4
	\item \textbf{Número de classes}: 3
\end{itemize}

Nesse problema um modelo físico de uma balança é avaliado. Na balança há dois diferentes pesos em diferentes distâncias do centro. Em cada configuração a balança pode estar pendendo para a esquerda, ou para a direita ou equilibrada. 