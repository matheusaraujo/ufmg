\documentclass{article}
\usepackage[utf8]{inputenc}
\usepackage[portuguese]{babel}
\usepackage{geometry}
\usepackage{graphicx}
\usepackage{mathtools}
\usepackage{amssymb}

\geometry{
 a4paper,
 total={170mm,257mm},
 left=20mm,
 top=20mm,
}

\newcommand{\ufmgdisciplina}{Organização Industrial para Engenharia}
\newcommand{\ufmgtema}{Prova 2}
\newcommand{\ufmgestudante}{Matheus Araujo}
\newcommand{\ufmgsemestre}{2018 01}
\newcommand{\ufmgprofessor}{Raoni Guerra Lucas Rajão}

\title{\ufmgdisciplina\,  - \ufmgtema \, - Resumo}
\author{\ufmgestudante}
\date{\ufmgsemestre}

\begin{document}

\maketitle

Material elaborado durante estudos para a \ufmgtema \,  da disciplina \emph{\ufmgdisciplina}. Professor \ufmgprofessor, UFMG, \ufmgsemestre.

\section{Unidade 3}

Objetivos

\begin{itemize}
    \item Definir os conceitos de gestão da qualidade, melhoramento e produtividade
    \item Esclarecer como esses conceitos se comportam em diferentes modelos de produção
    \item Analisar as ferramentas da gestão da qualidade no âmbito do modelo japonês de produção
\end{itemize}

\subsection{O Modelo Japonês}

\begin{itemize}
    \item \textbf{Sistema Toyota de Produção (STP)} - Principais práticas \textit{lean} para viabilizar a implantação do STP e as diferentes derivações do \textit{lean manufacturing} ou \textit{lean production}.
    \item Identificação do fluxo de valor a partir dos 5 passos
    \item Tipologia de desperdícios que podem incidir nos diversos processos organizacionais
    \item Práticas \textit{lean} para o combate e eliminação das fontes geradoras de desperdícios. Seu princípios , fundamentos e lógica de funcionamento
    \item Derivações da metodologia \textit{lean} e aplicações em contextos do desenvolvimento de produtos (\textit{lean product development}) e negócios (\textit{lean startup methodology})
\end{itemize}

\subsection{Melhoramento}

\begin{itemize}
    \item Atividade de melhoramento é fundamental para as organizações sobreviverem. Para tanto, esta atividade deve ser planejada e gerenciada.
    \item Fundamentalmente, existem dois tipos de melhoramento: melhoramento radical no qual a melhoria ocorre por meio de uma mudança de maior magnitude; e o melhoramento contínuo, no qual a melhoria ocorre de maneira incremental, por meio de pequenas e sucessivas mudanças.
    \item Para ser mais competitiva, a organização deve buscar conjuntamente o melhoramento radical e continuo. 
    \item As atividades de melhoramento contínuo almejam objetivos organizacionais alinhados com vários programas organizacionais que visam a melhoria da competitividade e eficiência da organização
    \item O melhoramento contínuo é conduzido segundo uma abordagem estruturada (Ciclo PDCA) e depende da adequada capacitação e aplicação de diversos métodos e técnicas.
    \item O melhoramento radical se faz necessário para lidar com problemas crônicos e de natureza interdepartamental.
    \item Cada atividade de melhoramento radical deve ser conduzida sobre a lógica de projetos, por meio de um procedimento disciplinado para resolução de problemas.
\end{itemize}

\subsection{Gestão da Inovação}

\begin{itemize}
    \item Inovação tecnológica não é um fenômeno recente: tem impactado o desempenho das empresas e as relações de mercado há séculos. No entanto, essa realidade tem se intensificado significativamente nas últimas décadas.
    \item Uma definição para inovação é: exploração de novas ideias com sucesso. Isso envolve novas tecnologias ou aplicações tecnológicas em produtos, serviços, processos de produção e modelos de negócio
    \item Entre o progresso da ciência e a inovação efetivamente adotada há vários níveis de análise. Ademais, nem toda inovação advém de conhecimento científico novo. Contudo, quando isso ocorre, o impacto potencial da inovação é bastante significativo
    \item Inovações podem se diferenciar por seus tipos (produto, processo...) e intensidades (incremental, radical...). Contudo, cada \textit{stakeholder} (usuário, reciclador, montador...) pode ter diferentes percepções sobre a inovação.
    \item A dinâmica da inovação carrega consigo características típicas de cada setor em que ocorre, o que também pode se transformar em oportunidades para empresas que tentam quebrar os vieses tradicionais de sua área de atuação.
    \item No contexto das organizações industriais, as inovações resultam de um processo de negócio. Modelos como o \textit{Stage-Gates} e \textit{Funil} são as formas mais comuns de representar a dinâmica deste processo.
    \item O termo "Inovação Aberta" remete à busca de fontes externas para inovação e pode ocorrer por meio de codesenvolvimento, inovação colaborativa, \textit{joint ventures} e modelos \textit{open-source}.
    \item Um sistema de gestão da inovação tem o processo de inovação como seu elemento central, mas agrega um conjunto de elementos gerenciais para que o processo ocorra de forma repetida na organização.
    \item Exemplos de modelos de referência para sistemas de gestão da inovação são o Modelo das Duas Rodas, o Sistema para Inovação Industrial do CIMS e o DNA.
    \item A Função Inovação (FI) se dá pela existência de uma equipe que se responsabilize por fazer a inovação acontecer no ambiente industrial. 
    \item Entre as atribuições típicas da FI estariam a gestão de ideias, realização de parcerias, gestão de propriedade intelectual e a busca de fomento.
\end{itemize}

\section{Unidade 4}

Objetivos

\begin{itemize}
    \item Apontar os conceitos desta unidade e levantar os elementos mais importantes para a produção e para o trabalho;
    \item Recordar os conceitos e aplicações discutidos nas unidades anteriores e relacionar com os tópicos desta unidade;
    \item Organizar os conceitos e propor discussões transversais e articuladas;
    \item Discutir a organização industrial dentro de uma perspectiva criativa e inovadora, reavaliando as aplicações propostas pela literatura para a organização da produção e do trabalho.
\end{itemize}

\subsection{Desenvolvimento sustentável}

\begin{itemize}
    \item O desenvolvimento sustentável é \textit{aquele que atende às necessidades do presente sem comprometer a possibilidade de as gerações futuras atenderem a suas próprias necessidades}, em uma sinergia entre os campos ecológico, social e econômico. 
    \item A ecoeficiência leva à disponibilização de bens e serviços a preços competitivos que satisfaçam as necessidades humanas e reduzam progressivamente o impacto ecológico e a intensidade de utilização de recursos naturais de forma compatível com a capacidade de renovação estimada para o planeta Terra 
    \item O grau de inovação necessário deve ser tal que as soluções propostas sejam intrinsecamente sustentáveis ou sejam novos cenários que correspondam ao estilo de vida sustentável, onde grandes mudanças técnicas e culturais são requeridas. 
    \item Para orientar a tomada de decisões relacionadas com aspectos de desenvolvimento ambiental nas empresas, várias ferramentas de gestão têm sido desenvolvidas: ferramentas para avaliação e análise, para ação, e para comunicação e relação com o público. 
    \item A ISO, ocupa-se na padronização de seis destas ferramentas, a conhecida série ISO 14000: Sistemas de Gestão Ambiental, Auditoria Ambiental, Avaliação da Performance Ambiental, Avaliação do Ciclo de Vida, Design para o Meio Ambiente, e Rotulagem Ambiental. 
    \item Os Sistemas de Gestão Ambiental (SGA), são focados principalmente em aspectos organizacionais e nos processos, levando à empresa a engajar-se de forma estruturada em uma estratégia de desenvolvimento sustentável. 
    \item A avaliação do ciclo de vida (ACV), é uma ferramenta quantitativa de avaliação ambiental, consiste basicamente em definir, medir e avaliar as implicações ambientais de todas as entradas (materiais e energia) e saídas (emissões no ar, solo e água) associadas ao produto. 
    \item A ISO divide a ACV em quatro etapas:
    \begin{itemize}
        \item Definição do objeto e escopo; 
        \item Inventário do ciclo de vida; 
        \item Avaliação do impacto do ciclo de vida; 
        \item Interpretação da avaliação do ciclo de vida
    \end{itemize}
    \item O ecodesign é uma prática que inclui, além das considerações ambientais do produto, aspectos ambientais no design dos processos, daí sua adequação a uma abordagem voltada para a adoção de princípios de desenvolvimento sustentável pela empresa.
    \item Os caminhos para a sustentabilidade são diversos, dada as diferentes abordagens e linhas de pensamento. Alguns percursos levam a soluções por intermédio da tecnologia, já outros conduzem a alternativas ligadas a redução do consumo.
\end{itemize}

\subsection{Gestão do Conhecimento}

\begin{itemize}
    \item Existem dois tipos de conhecimento: tácito e reificado. Assim, devemos pensar nos dois tipos de gestão.
    \item O conhecimento tácito é o conhecimento desenvolvido por meio da experiência, pela prática em alguma atividade.
    \item O conhecimento tácito pode ser: 
    \begin{itemize}
        \item somático (ser capaz de andar de bicicleta); 
        \item contingencial (pessoas que não se dão conta do seu conhecimento, mas são reconhecidas por fazerem algo que funciona); 
        \item e coletivo (exige um entendimento do contexto social e não é possível transformá-lo em código).
    \end{itemize}
    \item No conhecimento tácito coletivo, pessoas aculturadas desenvolvem habilidades que as tornam capazes de fazer julgamentos em suas práticas.
    \item Fazer julgamento é atribuir valor, de acordo com as convenções sociais que estão valendo naquele momento, sobre os aspectos percebidos do ambiente e na execução de uma atividade em curso
    \item O julgamento de similaridade e diferença é a capacidade de identificar contraste entre situações, ser capaz de perceber algo que está dentro ou fora de um intervalo tolerado, fazer aproximações, elaborar estimativas com base em experiências anteriores.
    \item O julgamento de relevância e irrelevância é a capacidade de atribuir valor a pessoas, objetos e eventos, dependendo da forma como elas se apresentam no contexto.
    \item O julgamento de risco e de oportunidade é a habilidade de avaliar consequências (de curto, médio e longo prazo) de ações ou eventos em curso ou futuros, de antecipar problemas ou acidentes, de decidir a hora certa de parar ou seguir em frente.
    \item O conceito de similaridade envolve analisar a experiência prática prévia do trabalhador em relação às habilidades requeridas pela nova atividade.
    \item A gestão do conhecimento reificado trata de conhecimento estruturado e disponibilizado para outras pessoas terem acesso por meio de um conjunto de atividades relacionadas à geração, codificação e transferência de informação.
    \item As tecnologias de informação e a eletrônica digital têm dado o suporte necessário para a organização, disseminação e recuperação de informações por meio de sistemas de informação.
    \item A ênfase da gestão do conhecimento reificado está na representação (ontologias) e na codificação e classificação (taxonomia) de material registrado (conteúdo) embutido em artefatos, estruturas, sistemas e repositórios.
\end{itemize}

\subsection{Indústria 4.0}

\begin{itemize}
    \item O mundo passa por mais um período de grande mudanças econômicas e sociais. Usufruindo de conquistas anteriores, a ênfase agora está na inteligência proporcionada à manufatura, com a informatização e conexão em rede de máquinas, aliada à automação industrial, abrindo um universo de possibilidades em diferentes setores
    \item Tal movimento está proporcionando novos modelos de negócio e descontinuando outros, bem como reformulando a produção, o consumo, os transportes, a educação, a saúde, os governos e as instituições.
    \item  Existem várias denominações relacionadas a esse movimento: Fábrica do Futuro, Fábrica inteligente, Manufatura Avançada, Internet Industrial, Internet das Coisas e Internet de Tudo são alguns deles, sendo mais utilizados os termos Quarta Revolução Industrial e Indústria 4.0, sendo o termo Indústria 4.0 o mais utilizado neste capítulo
    \item A Indústria 4.0 traz como principais características a descentralização de processos, a virtualização de sistemas e a versatilidade. Entre as principais vantagens, podem ser citadas: maior segurança de plantas industriais; menores custos em cadeias produtivas mais extensas, decorrentes da agilização de processos; melhor acompanhamento da manutenção de componentes; acompanhamento mais efetivo de linhas de produção; aplicação da manufatura enxuta e produção sob demanda.
    \item As tecnologias que estão ganhando mercado incluem simulações, integração de sistemas, Internet das Coisas, segurança cibernética, computação em nuvem, manufatura aditiva (3D), realidade aumentada, big data e robôs autônomos.
    \item Diversos setores estão tirando proveito da aplicação da Internet das Coisas, entre eles: automotivo, de energia (na geração, distribuição e consumo), na medicina e saúde, no comércio, agronegócio, transportes, construção civil (com casas e edifícios inteligentes).
    \item No Brasil, a Indústria 4.0 caminha ainda a passos lentos, sendo fundamental tomar ações para melhorar a formação e a capacitação da mão de obra para trabalhar em ambiente de alta tecnologia, reverter problemas de infraestrutura e desenvolver a capacidade de integrar inovações de ruptura à economia e sociedade
    \item Ser engenheiro requer combinar conhecimentos para solucionar problemas técnicos com os quais se defronta a sociedade. Em geral ele atua como parte de uma equipe tecnológica, que envolve vários perfis, e pode desempenhar funções em várias áreas, trabalhando em alguma especialidade da engenharia.
    \item Para ter sucesso como engenheiro no cenário atual, além de conhecimento técnico e domínio de programação, torna-se necessário: cultivar várias habilidades, entre elas a criatividade, a comunicação e o senso crítico; saber trabalhar em equipe, de forma proativa e ética; adaptar, dar forma e aproveitar o potencial das mudanças pela aplicação dos quatro tipos de inteligência (contextual, emocional, inspirada e física).
    \item O ensino da engenharia precisa evoluir para suprir a lacuna identificada entre as demandas de engenharia científica e prática. Uma das iniciativas que tem ganhado espaço entre as universidades é o padrão educacional CDIO - Conceiving-(Conceber), Designing-(Projetar), Implementing-(Implementar) e Operating-(Operar).
\end{itemize}

\end{document}

