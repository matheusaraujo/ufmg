\documentclass{article}
\usepackage[utf8]{inputenc}
\usepackage{geometry}

\title{Organização Industrial para Engenharia - Prova 1 - Resumo}
\author{Matheus Araujo}
\date{2018 01}

\geometry{
 a4paper,
 total={170mm,257mm},
 left=20mm,
 top=20mm,
}

\begin{document}

\maketitle

Material elaborado durante estudos para a Prova 2 da disciplina \emph{Organização Industrial para Engenharia}. Professor Raoni Guerra Lucas Rajão, UFMG, 2018 01.

\section{Unidade 1}

Objetivos

\begin{itemize}

    \item Apontar os principais elementos que incorporam e diferenciam a \emph{função produção} da \emph{função operações}, definindo como esses elementos constituem as estratégias organizacionais;
    
    \item Explicar como o contexto industrial brasileiro influencia nas estratégias industriais e sistemas produtivos estudados.
    
    \item Reunir e organizar as informações acerca das estratégias de operações, avaliando as influências na produção em geral.

\end{itemize}

\subsection{Função Produção e Operações}

\begin{itemize}

    \item A função produção é responsável pela produção e entrega de produtos e serviços aos consumidores.
    \item A função produção é uma função central, assim como o desenvolvimento de produtos e o marketing. Finanças e recursos humanos são funçoes de apoio.
    \item Modelo da função produção:
    \begin{itemize} 
        \item Recursos de entrada: materiais, informações e consumidores
        \item Processamento ou transformação 
        \item Saídas do processo: produtos e serviços
    \end{itemize}
    \item As operações se diferem nas dimensões:
    \begin{itemize}
        \item Volume
        \item Variedade
        \item Variação de demanda
        \item Visibilidade
    \end{itemize}
    \item Os objetivos de desempenho da produção são:
    \begin{itemize}
        \item Qualidade
        \item Rapidez
        \item Confiabilidade
        \item Flexibilidade
        \item Custo
    \end{itemize}
    \item Sistemas de produção podem ser: em massa, em lotes ou sob encomenda. 
    \begin{itemize}
        \item Cada sistema deve ter um layout que viabilize sua estratégia de produção e facilite os fluxos de materiais, equipamentos e pessoas
    \end{itemize}
    \item O balanceamento do trabalho consiste em projetar os tempos das atividades dos pontos de trabalho alinhado ao tempo de ciclo.
    \item A previsão de demanda envolve o longo e o médio prazo.
    \begin{itemize}
        \item No curto prazo tem-se o planejamento de demanda.
    \end{itemize}
    \item O planejamento de produção envolve o longo e o médio prazo.
    \begin{itemize}
        \item No curto prazo tem-se a programação da produção.
    \end{itemize}
    \item A produção empurrada é baseada no planejamento da demanda.
    \item A produção puxada é usada para atender pedidos em carteira.
    \item O Gerente de Produção tem as responsabilidades:
    \begin{itemize}
        \item Entender os objetivos estratégicos da produção
        \item Entender as medidas de desempenho
        \item Se envolver nas atividades de projeto, planejamento e controle e da melhoria do processo
        \item Dominar tecnologias de produção
        \item Dominar a gestão do conhecimento
        \item Proteger o meio ambiente
        \item Ser responsável socialmente
    \end{itemize}
\end{itemize}

\subsection{Estratégia Organizacional}

\begin{itemize}
    \item Estratégia organizacional pode ser pensada como a navegação de uma organização em um mapa ao longo do tempo.
    \item Três elementos básicos compõem a noção de estratégia organizacional:
    \begin{itemize}
        \item Origem, \emph{de onde}
        \item Rota, \emph{por onde}
        \item Destino, \emph{para onde}
    \end{itemize}
    \item Paradigmas de compreensão da estratégia organizacional:
    \begin{itemize}
        \item \emph{Estratégia-como-processo} - enfatiza o mapeamento
        \item \emph{Estratégia-como-conteúdo} - enfatiza o mapa
        \item \emph{Estratégia-como-prática} - enfatiza a navegação em si
    \end{itemize}
    \item O mapeamento de estratégias pode ser feito a partir de um processo prévio de planejamento estratégico (estruturado em passos sequenciais) ou incrementalmente, à medida que cada passo é dado. Além disso, múltiplos cenários podem ser levados em consideração para lidar com a incerteza que limita a nossa racionalidade nesse processo.
    \item Na estratégia-como-conteúdo, tenta-se abstrair, dos mapas organizacionais, fatores atemporais que expliquem a vantagem competitiva sustentável das organizações de sucesso.
    \item No contexto desse paradigma, quatro perspectivas principais foram desenvolvidas: 
    \begin{itemize}
        \item A visão baseada em setores industriais, focada em fatores relacionados à estrutura e à dinâmica das indústrias; 
        \item A visão baseada em recursos, com enfoque na formação de competências essenciais internas à organização; 
        \item A visão baseada em instituições, que destaca o macroambiente em suas múltiplas dimensões e na sua influência sobre a internacionalização das organizações; 
        \item A visão baseada em estruturas organizacionais, focada nas diversas formas que uma organização pode tomar, tendo em vista os seus vários \textit{shareholders}, \textit{stakeholders} e administradores.
    \end{itemize}
    \item No paradigma da estratégia-como-prática, critica-se a racionalização da estratégia, propondo-se um retorno à observação da estratégia-na-prática, nos seus contextos históricos específicos, à medida que a navegação organizacional de fato emerge da miríade de atividades dos atores envolvidos.
\end{itemize}

\section{Unidade 2}

Objetivos

\begin{itemize}
    \item Enunciar os principais modelos de produção discutidos, sublinhando as principais aproximações ou distanciamentos entre os mesmos.
    \item Esclarecer cada modelo a partir das modificações na produção, tecnologia e trabalho, ilustrando com exemplos.
    \item Debater as características dos modelos sob um prisma crítico, voltado para as influências no ambiente de trabalho e na condição do trabalhador.
    \item Formular hipóteses e avaliar a aplicação (ou não) das metodologias de cada modelo em diversos ambientes organizacionais e em diversos sistemas produtivos.
\end{itemize}

\subsection{A Escola Sócio-Ténica}

\begin{itemize}
    \item A discussão sobre a escola sócio-técnica é uma tentativa de superar aspectos do modelo Taylorista/Fordista, no que tange ao modelo prescrito e especializado de produção e trabalho. Originou na década de 1950, a partir de estudos realizados no \emph{Tavistock Institute of Human Relations}, Inglaterra. Trata-se de uma instituição voltada para estudos de comportamento e grupos no trabalho. As pesquisas resultaram em importantes conceitos para uma reorganização do processo produtivo. 
    \item A partir de 1940, as técnicas tayloristas de produção foram inseridas nas minas de carvão, na Inglaterra: adotou-se a mecanização do trabalho, em um novo método de extração de carvão, no qual os trabalhadores foram separados em tarefas especializadas. Ao observar alguns problemas, os pesquisadores buscaram uma alternativa para a organização do trabalho. Para tanto, observaram outra experiência de trabalho nas minas, onde as atividades e os trabalhadores eram organizados em subgrupos interdependentes, relativamente autônomos e recebiam o mesmo salário, de acordo com a produção do grupo. A partir de 1950, foram desenvolvidas e difundidas as ideias acerca de uma forma de trabalho e organização alternativa ao modelo de mecanização e fragmentação das tarefas, unindo elementos sociais, técnicos e psicológicos do trabalho, delineando a abordagem da escola sócio-técnica.
    \item Trata-se de uma forma sistêmica de pensar o trabalho, o qual está inserido em um ambiente, tanto organizacional quanto externo. Portanto, o processo de trabalho deve ser planejado considerando essas influências e intercâmbios ambientais. Esse enfoque sistêmico está fortemente embasado nos conceitos advindos das discussões acerca da Teoria Geral dos Sistemas.
    \item Conceitos fundamentais: unidade básica de trabalho, grupos de trabalho, autoregulação do trabalho, diversificação, autonomia e partes complementares. Princípios básicos: compatibilidade, mínima especificação crítica e controle de variâncias.
    \item O conceito de indivíduo e grupos inerente à abordagem sócio-técnica é baseado nas discussões da ciência comportamental e psicologia social. O indivíduo possui um inconsciente, que rege seus instintos e um consciente que governa sua relação com o ambiente externo. De acordo com a organização desses elementos, cada indivíduo apresentará diferentes expectativas com relação ao seu trabalho, influenciando no projeto organizacional e nos resultados produtivos.
    \item Algumas etapas da organização do processo produtivo, segundo essa abordagem, para definir que parâmetros precisam ser reforçados ou implementados: avaliação, Identificação das operações fundamentais de operação, identificação das variações fundamentais, da análise do sistema social, bem como do ambiente externo da organização.
    \item Com a realização dessas etapas subsidia-se a implementação de mudanças no processo de organização do trabalho. Duas mudanças práticas na organização do trabalho podem ser ressaltadas na abordagem sócio-técnica: o trabalho em docas de produção (sistema técnico) e a implementação dos grupos semi-autônomos (sistema social). A produção em docas substitui os arranjos tradicionais em linha, baseados na perspectiva taylorista. Nesse sistema, o produto fica imóvel enquanto os trabalhadores trabalham na sua montagem, aproximando-se, quando necessário, do produto, com uso de equipamentos universais. Para o funcionamento das docas, o trabalho deve ser organizado em grupos. Nos grupos semi-autônomos, o trabalho é realizado com autonomia para tomada de decisões em relação as tarefas e responsabilidades coletivas. Dessa forma, contribui-se para que as pessoas se relacionem dentro do grupo, e busquem atingir objetivos comuns, com máxima eficiência.
    \item A arquitetura organizacional é outro elemento essencial para o desempenho dos GSA. A organização deve ser um ambiente de captação e desenvolvimento de forças e competências requeridas para o trabalho em grupo. Deve promover a comunicação e incentivar a autonomia, bem como o trabalho multifuncional dos especialistas no núcleo operacional, voltado para a resolução dos problemas da produção e dos projetos, ou seja, deve ser uma organização qualificante.
    \item Aplicação dos princípios e conceitos da escola sócio-técnica a partir da experiência no setor automobilístico, na Volvo: as mudanças ocorridas em Uddevalla mesclaram uma retomada do trabalho menos especializado com a introdução de equipamentos e inovações tecnológicas no ambiente de produção. Muitas dessas inovações já tinham sido testadas na unidade fabril Kalmar, em 1974.
    \item No Brasil, as ideias das escola sócio-técnica foram implementadas em algumas empresas a partir de 1990, baseadas em revisões críticas teóricas, caracterizando uma abordagem moderna do modelo sócio-técnico. Essa abordagem ressalta a inter-relação da organização do trabalho com o gerenciamento da produção. No entanto, é válido sublinhar que não se trata de um modelo produtivo disseminado nas indústrias brasileiras, mesmo diante de uma crescente preocupação com os aspectos sociais unidos à produtividade, flexibilização e inovação. A aplicação dos princípios sóciotécnicos no contexto brasileiro ainda é incipiente, mais voltada para a indústria de processos contínuos.
    \item Diante dos limites e desafios apresentados, o engenheiro exerce papel central na organização do trabalho, considerando a abordagem sócio-técnica, visto que é o agente de coordenação e interligação entre os processos de definição da esquemática do trabalho, e o uso da tecnologia. O engenheiro contemporâneo precisa compreender a escola sócio-técnica para além de sua concepção tradicional. Deve olhar para a abordagem sócio-técnica como uma nova forma de organizar a empresa estrategicamente, pensando no próprio contexto social, ambiental e político da mesma.    
\end{itemize}

\subsection{A Escola Sócio-Ténica}

\begin{itemize}
    \item A flexibilidade tem sido uma importante fonte de vantagem competitiva para as organizações.
    \item Flexibilidade diz respeito a mudanças de estado do sistema produtivo; ao tempo que se leva para a mudança de estado; e à manutenção ou melhoria de desempenho durante e após o processo de mudança.
    \item Há vários tipos de flexibilidade: extra-empresa, estratégica, de volume, de mix, de gama, para operações sazonais, para suportar mau funcionamento do sistema produtivo, para suportar erros de previsão
    \item A flexibilidade pode ser alcançada por meio de mudanças nas relações entre empresas, na estrutura organizacional da empresa, na organização da produção, na organização do trabalho, além de mudanças na concepção dos produtos
    \item Formas organizacionais mais fluidas são mais convenientes para a obtenção de flexibilidade, ao contrário de organizações burocratizadas, extremamente formalizadas
    \item Estruturas matriciais, organização por projetos, arranjo físico celular, organização do trabalho em equipes e que privilegie o desenvolvimento de competências no trabalho são práticas reconhecidas como provedoras de flexibilidade para as organizações.
    \item A flexibilidade possui também aspectos negativos, podendo ter impactos negativos sobre o meio ambiente, sobre os custos e, sobretudo, sobre a saúde dos trabalhadores, como a sobrecarga de trabalho.     
\end{itemize}

\end{document}

